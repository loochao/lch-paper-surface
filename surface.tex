\documentclass[twocolumn,showpacs,preprintnumbers,amsmath,amssymb]{revtex4}

\usepackage{graphicx}  % Include figure files
\usepackage{dcolumn}   % Align table columns on decimal point
\usepackage{bm}        % bold math
\usepackage{braket}
\begin{document}

\preprint{APS/123-QED}

\title{A Phenomenological Model of Photoinduced\\ Surface Relief
  Formation in Arsenic Sulfide}
% Force line breaks with \\

\author{Chao Lu}
\author{Daniel Recht}
\author{Craig Arnold}
\email{cbarnold@princeton.edu}
\affiliation{Princeton University}
\date{\today}

\begin{abstract}
Chalcogenide materials, such as Arsenic Sulfide (As$_2$S$_3$), are
well-known for their sensitivities to external stimuli, especially
illumination. Among various photo-induced phenomena, this paper
focused on mass transport issues. Thin films of As$_2$S$_3$ respond to
non-uniform illumination by changing their surface morphologies. The
volume expansion can be as large as 0.05\% with appropriate
illumination. However, there is still much debate on the mechanism for
this process. We present a new model describing the surface structural
modification by employing the incompressible fluid equation, driven by
a photo-induced pressure. This equation, derived without any
consumption of the microstructure of As$_2$S$_3$, is shown to be
sufficient and effective in predicting many aspects of photo-induced
surface relief formation.
\end{abstract}

\pacs{Valid PACS appear here}    %PACS, the Physics and Astronomy Classification Scheme.
\keywords{Chalcogenide, Photo-induced phenomena, modeling}
\maketitle

\section{\label{sec:intro}Introduction}

It has long been known that photoinduced processes can lead to mass
transport in a variety of systems generating surface relief.  A
variety of systems exhibit photoinduced expansion and or contraction.

While it has long been known that thin arsenic sulfide
(As$_{2}$S$_{3}$) films expand when exposed to above-bandgap light,
the mechanism by which this occurs is still open to debate
\cite{igo74, hegedus, ganjoo}. Two fairly recent discoveries reveal
that the situation is far more complex than it first appears. First,
in 1994 Hisakuni and Tanaka found that exposure of As$_{2}$S$_{3}$ to
laser light could induce expansions a full order of magnitude greater
than had previously been observed even for very low intensities
\cite{hisakuni94}. In order to explain this athermal effect, Hisakuni
and Tanaka proposed and soon verified that As$_{2}$S$_{3}$ displays
photoinduced fluidity \cite{hisakuni95}.

The second major development was the discovery by Saliminia et
al. (hereafter Saliminia) in 2000 that thin films of As$_{2}$S$_{3}$
will respond to the polarization of a Gaussian beam to which they are
exposed by undergoing mass transport along the polarization vector
\cite{saliminia}. Salminia's group also exposed films to several
interference patterns with intensity gradients and to others with
modulated polarization and observed the formation of surface relief
gratings of a size comparable to the ``giant'' photoexpansion in both
cases. Although surface relief gratings had been observed in the past,
those early studies reported much smaller features \cite{galstyan}. In
2005, Asatryan et al.  (hereafter Asatryan) repeated Saliminia's
experiment at much lower intensities and found a similar result for
polarization modulation but did not observe grating formation under
intensity contrast \cite{asatryan05}. Saliminia and Asatryan's results
are summarized in table \ref{tab:comparison}.

\begin{table}
\begin{ruledtabular}
\begin{tabular}{l l l l }
\multicolumn{2}{c}{\textbf{Illumination Conditions}}&\multicolumn{2}{c}{\textbf{Surface Relief}}\\
\hline
\textbf{Polarization of}&\textbf{Modulated}&\textbf{Seen by}&\textbf{Seen by}\\
\textbf{Initial Beams}& \textbf{Quantity}&\textbf{Saliminia}&\textbf{Asatryan}\\
\hline
s-s &Intensity&$\frac{1}{2}$ p-p    &   None\\
p-p &Intensity&Major    &   None\\
s-p&Polarization&$\frac{3}{4}$ p-p&None\\
45-135&Polarization&$\approx$ p-p&Major\\
RCP-LCP&Polarization&Not tested&Major\\
\end{tabular}
\end{ruledtabular}
\caption{A comparison of the observations reported by Saliminia and
Asatryan concerning which illumination conditions lead to the
formation of surface relief gratings. See figure \ref{fig:setup} for a
definition of s and p polarization in terms of the film's natural
coordinate system. The angles in 45-135 polarization are measured
above the positive $x$ (s) axis. LCP and RCP refer respectively to
left and right circular polarization.} \label{tab:comparison}
\end{table}

After addressing and explaining the apparent contradictions between
Saliminia and Asatryan's data, this paper presents a phenomenological
model describing surface relief formation as incompressible flow
driven by a photoinduced pressure and damped by surface
tension. Despite the fact that both polarization and intensity
modulation may lead to this effect, a single expression for the
photoinduced pressure, derived without any assumptions about the
microstructure of As$_2$S$_3$, is shown to be sufficient. Finally,
results of the model are considered.

\begin{figure}[!htbp]
  \includegraphics[width=2.45in]{figure/sppic.png}
  \caption{Schematic diagram of the experimental setup used by Saliminia and Asatryan.
                                Two beams, W1 and W2, interfere on the surface of a thin film of As$_{2}$S$_{3}$ leading
                                to the pictured definitions of the coordinate system and polarization directions.}
  \label{fig:setup}
\end{figure}

\section{The Model}

\subsection{Reconciling Saliminia and Asatryan}
\label{sec:chalcmod}
The most important issue raised by table \ref{tab:comparison} is that
Saliminia observed grating formation in response to intensity
modulation while Asatryan did not. Naturally, one must look to their
procedures for an answer. The most striking difference between their
two studies is the intensity of the light used to illuminate the
As$_{2}$S$_{3}$ films. While Saliminia used an intensity of $105$
W/cm$^{2}$, Asatryan used just $350$ mW/cm$^{2}$. One might be tempted
to suggest that 1) photoinduced fluidity is a necessary precursor to
mass transport and 2) Asatryan's intensity was insufficient to bring
this about. Unfortunately for this explanation, Trunov has shown
conclusively that the intensity used by Asatryan would be more than
enough to induce fluidity in As$_{2}$S$_{3}$ \cite{trunov03}.

A close reading of Asatryan and Saliminia reveals another important,
though easily overlooked discrepancy between their methods. In
preparing films, Asatryan chose to expose them to a large fluence of
circularly polarized light before testing for surface relief
formation. As justification for this, she cites Lyubin's classic study
of scalar and vector photoinduced phenomena in arsenic
chalcogenides. While this paper does indeed report that scalar-vector
pairs such as photoinduced changes in refractive index (a scalar
effect) and photoinduced birefringence (its vector analogue) appear to
operate according to different mechanisms, it does not imply that
those mechanisms are completely independent as Asatryan seems to
assume \cite{lyubin89}. Nor can one take for granted, as Asatryan does
by citing Lyubin, that all scalar and vector effects are due to the
same underlying cause. These assumptions manifest themselves in
Asatryan's neglect of the possibility, not addressed by Lyubin, that
anisotropic activation of a scalar mechanism can produce vector
results. This seems to be precisely the case observed by Saliminia.
Specifically, Asatryan's saturation procedure corresponds to a larger
scale version of Saliminia's exposure of a film to a circularly
polarized Gaussian beam, a process that caused surface relief. Thus,
Asatryan's pre-exposure could easily have saturated the intensity
effect observed by Saliminia. It is therefore no wonder that Asatryan
did not observe surface relief due to intensity modulation. However,
the fact that Asatryan did manage to observe grating formation due to
polarization modulation despite this saturation supports the existence
of independent scalar and vector mass transport effects. In short,
contrary to Asatryan's own assertions, there is no conflict between
her results, Saliminia's study, and Lyubin's principles. In terms of
the model being constructed, this discussion suggests that any
expressions for the pressure will need to contain two independent
components: one due to the vector effect and one due to its scalar
counterpart.

\subsection{Fluid Dynamics of Surface Relief Formation}

\label{sec:fluids} Stipulating laminar flow of the glass and
time-independent illumination that varies in one direction along the
surface (the $x$ axis) but is uniform along the other surface axis
($y$) and with depth ($z$) brings surface relief formation in
As$_2$S$_3$ within the scope of the Navier-Stokes equation simplified
into a two-dimensional boundary layer equation in $x$ and $z$
\cite{levich}. The coordinate system thus described is shown in figure
\ref{fig:setup}.

\begin{equation}
\frac{\partial v_x}{\partial t}+v_x\frac{\partial v_x}{\partial x} +v_z\frac{\partial
v_x}{\partial z} = - \frac{1}{\rho}\frac{\partial \mathcal{P}}{\partial
x}+\nu\frac{\partial^2 v_x}{\partial z^2}+f \mathrm{,} \label{eq:levstokes}
\end{equation}

in which the $v_i$'s are components of the velocity vector, $\rho$ is
the mass density, $\mathcal{P}$ is the total pressure, $f$ is the body
force, and $\nu$ is the kinematic viscosity. % Applying a thin-film
approximation sanctions the replacement of $\mathcal{P}$ with its
value at the surface since there is very little depth over which the
pressure can change. At the surface, $\mathcal{P}$ comprises surface
tension, $\mathcal{S}$, and the photoinduced pressure, $P$. Surface
tension is traditionally taken to be proportional to surface
curvature. Symbolically

\begin{equation}
\mathcal{S}= \sigma \frac{\frac{d^2h}{dx^2}}{\left[1+\
\left(\frac{dh}{dx}\right)^2\right]^{3/2}}\approx \sigma \frac{d^2h}{dx^2} \mathrm{,}
\label{eq:surften}
\end{equation}

where $\sigma$ is a constant with units of force per length, $h$ is
the (spatially varying) thickness of the film, and the last step is
justified by the thin film approximation since this condition implies
$dh/dx\ll 1$. The photoinduced pressure, which is simply assumed to
exist, is discussed at length in Section \ref{sec:presquant}. For now
it is enough to note that since the illumination is modulated only
along the $x$ axis, the photoinduced pressure can vary only with
$x$. The thin film approximation thus makes it possible to say that

\begin{equation}
\mathcal{P} \approx P(x)-\sigma\frac{\partial^2 h}{\partial x^2} \mathrm{.}
\end{equation}

In addition, there are no body forces to speak of so $f=0$.

Combining this information with equation \ref{eq:levstokes} leads to
\begin{equation}
  \frac{\partial v_x}{\partial t}+v_x\frac{\partial v_x}{\partial x} +v_z\frac{\partial
                                v_x}{\partial z} = - \frac{1}{\rho}\frac{\partial}{\partial
                                x}\left[P(x)-\sigma\frac{\partial^2 h}{\partial x^2}\right]+\nu\frac{\partial^2
                                v_x}{\partial z^2} \mathrm{.} \label{eq:lastfluid}
\end{equation}
The thin film approximation also implies that
\begin{equation}
v_x\frac{\partial v_x}{\partial x} \ll v_z\frac{\partial v_x}{\partial z}
\end{equation}
since $v_x$ and $v_z$ are of roughly the same order and the film is
assumed to be much wider than it is thick. That accounts for one term
in equation \ref{eq:lastfluid}, but more can be said. Following the
analyses of Ledoyen et al. and Pimputkar et al., it is possible to
drop all the terms on the right-hand side because they turn out to be
small in practice \cite{ledoyen, pimputkar, barrett}. Doing so yields
\begin{equation}
\frac{\partial^2 v_x}{\partial z^2} \approx \frac{1}{\eta}\left[\frac{\partial
P(x)}{\partial x}-\sigma\frac{\partial^3 h}{\partial x^3}\right] \mathrm{,}
\label{eq:start}
\end{equation}
where $\eta= \rho\nu$ is the dynamic viscosity.

Equation \ref{eq:start} appears to be solvable. Hence, it is time to
compile a list of constraints and boundary conditions. The first of
these is the continuity equation
\begin{equation}
\frac{\partial v_x}{\partial x}+\frac{\partial v_z}{\partial z} =0 \mathrm{,}
\label{eq:continuity}
\end{equation}
which is derived from incompressibility and conservation of
mass. Next, assuming perfect adhesion to the substrate implies
\begin{equation}
v_x=v_z=0 \mathrm{\ at\ } z=0 \mathrm{,} \label{eq:substrate}
\end{equation}
where $z=0$ is the film-substrate interface. At the free surface of
the film, the shear stress along $z$ goes to zero \cite{levich,
barrett}. Symbolically,
\begin{equation}
\frac{\partial v_x}{\partial z}=0 \mathrm{\ at\ } z=h \mathrm{.} \label{eq:shear}
\end{equation}
Finally, the $z$ velocity at the free surface is the rate of change in
the height. This can be represented formally as
\begin{equation}
v_z=\frac{\partial h}{\partial t} \mathrm{\ at\ } z=h \mathrm{.} \label{eq:vzbound}
\end{equation}
As will become evident, equations \ref{eq:continuity} through
\ref{eq:vzbound} are enough to specify the problem.

Since the right-hand side of equation \ref{eq:start} has no $z$
dependence, the whole equation can be integrated with respect to $z$.
\begin{equation}
\frac{\partial v_x}{\partial z}  \approx \frac{z}{\eta}\left[\frac{\partial
P(x)}{\partial x}-\sigma\frac{\partial^3 h}{\partial x^3}\right]+C_1 \mathrm{,}
\end{equation}
where $C_1$ is an integration constant. Applying the shear stress
boundary condition (equation \ref{eq:shear}) allows for the
determination of $C_1$. Thus
\begin{equation}
\frac{\partial v_x}{\partial z} \approx \frac{\left(z-h\right)}{\eta}\left[\frac{\partial
P(x)}{\partial x}-\sigma\frac{\partial^3 h}{\partial x^3}\right] \mathrm{.}
\end{equation}
Integrating with respect to $z$ again gives
\begin{equation}
v_x \approx \frac{\left(z^2/2-hz\right)}{\eta}\left[\frac{\partial P(x)}{\partial
x}-\sigma\frac{\partial^3 h}{\partial x^3}\right]+C_2 \mathrm{.}
\end{equation}
Application of the $x$ part of the substrate boundary condition
(equation \ref{eq:substrate}) clearly shows that $C_2$ is 0. Taking
the derivative of both sides with respect to $x$ and applying the
continuity condition (equation \ref{eq:continuity}) yields
\begin{equation}
-\frac{\partial v_z}{\partial z}  \approx \frac{\partial}{\partial x}
\left(\frac{\left(z^2/2-hz\right)}{\eta}\left[\frac{\partial P(x)}{\partial
x}-\sigma\frac{\partial^3 h}{\partial x^3}\right]\right) \mathrm{.}
\end{equation}
This too can be integrated with respect to $z$.
\begin{equation}
-v_z \approx \frac{\partial}{\partial
x}\left(\frac{\left(z^3/6-hz^2/2\right)}{\eta}\left[\frac{\partial P(x)}{\partial
x}-\sigma\frac{\partial^3 h}{\partial x^3}\right]\right)+C_3 \label{eq:almost}
\end{equation}
The $z$ part of the substrate boundary condition reveals $C_3$ to be $0$ as well.

Finally, setting $z = h$ and applying the last boundary condition (equation
\ref{eq:vzbound}) gives
\begin{equation}\
\frac{\partial h}{\partial t} \approx \frac{\partial}{\partial
x}\left(\frac{h^3}{3\eta}\left[\frac{\partial P(x)}{\partial x}-\sigma\frac{\partial^3
h}{\partial x^3}\right]\right) \mathrm{.} \label{eq:last}
\end{equation}

Given $P(x)$, equation \ref{eq:last} is readily solvable by standard
numerical methods.  Accordingly, the final component of this model is
a suitable expression for this function. %as will be seen in
                                                                                                                                                                                                                                                                %Section\ref{sec:imp}.
Before accepting it and moving on, however, prudence
recommends the application of some physical intuition. Equation
\ref{eq:last} gives the spatial dependence of the rate and direction
of surface relief formation. It seems reasonable that this rate be
inversely proportional to the viscosity of the film. Furthermore,
noting that the pressure is some function of the electric field, the
appearance of its gradient (which must depend, at least in part, on
the electric field gradient) is reminiscent of Saliminia's rough model
discussed in Section \ref{sec:chalcmod} \cite{saliminia}. Naively
speaking, this means that shortening the period of electric field
variation should increase the magnitude of the surface relief growth
rate. This effect is then countered by the corresponding increase in
the surface tension since shorter periods have more curvature. All in
all, the picture of balance thus painted is quite believable.

\subsection{Quantifying the Pressure}
\label{sec:presquant}

In general, the photoinduced pressure can depend anisotropically on
the electric field of the incident light. Symbolically,
$P=P(E_x,E_y)$. This function can be expanded in the complex
components $E_x$ and $E_y$ (see figure \ref{fig:setup}) according to
\begin{equation}
P(E_x,E_y)=a_1E_x+a_2E_y+a_3E_x^2+a_4E_xE_y+a_5E_y^2+\mathcal{O}(3) \mathrm{,}
\label{eq:impress}
\end{equation}
where the $a_i$'s are real constants. Absent from equation
\ref{eq:impress} is a provision to ensure that the pressure is real.
Two choices can be thrown away immediately: taking the real part of
the entire right-hand side of equation \ref{eq:impress} and saying
that $P(E_x,E_y)=P(|E_x|,|E_y|)$. The former leads to pressures that
oscillate rapidly in time about a mean of 0. The latter ignores all
phase information and thus cannot hope to explain s-p interference.

The appropriate choice is to require that
$P(E_x,E_y)=P(\tilde{E_x},\tilde{E_y})$ where
$\tilde{E_x}=\Re\mathrm{e}\left\{E_x\right\}$ \cite{recht06}. In this
case,
\ref{eq:impress} becomes
\begin{equation}
P(E_x,E_y)=a_1\tilde{E_x}+a_2\tilde{E_y}+a_3\tilde{E_x}^2+a_4\tilde{E_x}\tilde{E_y}+a_5\tilde{E_y}^2+\mathcal{O}(3)
\mathrm{.} \label{eq:repress}
\end{equation}
Now, the electric field oscillates so quickly that one could not
reasonably expect the As$_2$S$_3$ to respond to it in other than a
time-averaged manner. Since the time averages of $\tilde{E_x}$ and
$\tilde{E_y}$ are both 0, they can be dropped from equation
\ref{eq:repress}. Thus, ignoring third and higher order terms
\begin{equation}
P(E_x,E_y) \approx \left\langle
a_3\tilde{E_x}^2+a_4\tilde{E_x}\tilde{E_y}+a_5\tilde{E_y}^2\right\rangle \mathrm{,}
\label{eq:repress2}
\end{equation}
where angular brackets indicate time averaging.

Further consideration of the pressure arises from its physical origin
(see figure \ref{fig:diffusion} and \ref{fig:realign}). The
diffusion units, which are size of medium order range (up to 2
coordination sphere), are simplified to be the electrical-induced
dipoles. The strength of the dipoles could be calculated by equation
$\vec{P} = \int d^3 r \rho(\vec{r}) \vec{r}$, where the charge
distribution $\rho(\vec{r})$ could be obtained from quantum
calculation with high accuracy. At the present of optical radiation,
the weak connections in-between the diffusion units, which is the van
der waals force, are broken, then these units are free to move, so our
treatment using Navier-Stokes equations to describe the diffusion
process is appropriate. Furthermore, the electrical field of the light
modify the diffusion units to be induced dipoles. Those dipoles starts
to align and move according to the polarization of electrical field,
until the balance is achieved, where new connection, i.e. van der
waals interactions are built up again. It is by this way, the units
diffuses, leading to the surface morphology modified by the optical
field.

\begin{figure}[!htbp]
  \includegraphics[width=3.5in]{figure/dipole.png}
  \caption{The diffusion units are assumed to be optical-induced
                                dipoles. The figure on the left is a SEM observation of the
                                layered structure of As$_2$S$_3$; those layers are diffusion
                                units.}
  \label{fig:diffusion}
\end{figure}

\begin{figure}[!htbp]
  \includegraphics[width=3.5in]{figure/diffusion.png}
  \caption{Dipole realignment under electrical field causes diffusion,
                                leading to the volume expansion.}
  \label{fig:realign}
\end{figure}

Now we are able to calculate the photo-induced pressure
theoretically. Since pressure is always proportional to energy density
so we start with relation $Pressure \propto \partial Energy / \partial
V$. Equations describing the total free energy density of such dipole
interaction system in present of the electric field are present in
Landau's textbook \cite{Landau},

\begin{equation}
  Energy = F_0 + \epsilon_{ik}\epsilon_0E_iE_k \mathrm{.}
  \label{landau}
\end{equation}

Where $F_0$ is the free energy of the system in absence of an external
field. In the dipole interaction model, $\epsilon_{ik}$ actually
describes the polarizability of diffusion units when exposed to
electric field of the light. Rigorous result for the pressure is
obtained from equation \ref{landau}, in terms of the stress tensor, which is
taken from the Landau's textbook \cite{Landau},

\begin{equation}
  \label{eq:tensor}
  \sigma_{ik} = \epsilon_0 E_i D_k \mathrm{.}
\end{equation}

Plugging $D_k = \Sigma (\epsilon_{km} E_m)$ into equation \ref{eq:tensor},
the stress tensor is simplified as

\begin{equation}
  \sigma_{ik} = \epsilon_0 E_i (\epsilon_{kx}E_x + \epsilon_{ky}E_y +
  \epsilon_{kz}E_z) \mathrm{.}
\end{equation}

Taking trace of the stress tensor then applying the time average,
the pressure is written as
\begin{equation}
  \label{eq:pressure1}
  \begin{split}
                                P &= \frac{1}{3}\braket{Trace(\sigma)} \\
                                &= \frac{1}{3}\braket{\sigma_{xx} + \sigma_{yy} + \sigma_{zz}} \\
                                &= \frac{\epsilon_0}{3}\braket{E_x(\epsilon_{xx}E_x + \epsilon_{xy}E_y) + E_y(\epsilon_{yx}E_x
                                  + \epsilon_{yy}E_y)}\\
                                &= \frac{\epsilon_0}{3}\braket{\epsilon_{xx}E_x^2 + 2
                                  \epsilon_{xy}E_xE_y + \epsilon_{yy}E_y^2} \mathrm{.}\\
  \end{split}
\end{equation}

% \begin{equation}
%   \label{eq:polarize}
%   \begin{split}
%     Pressure &\propto \partial Energy / \partial V\\
%     &= \braket{\frac{\partial (\epsilon_x^2 E_x^2 +
%     \epsilon_x \epsilon_y E_x E_y + \epsilon_y^2
%     E_y^2)}{\partial V}}\\
%     &= \braket{\frac{\partial \epsilon_x^2}{\partial V} E_x^2 + \frac{\partial \epsilon_x
%     \epsilon_y}{\partial V} E_x E_y + \frac{\partial \epsilon_y^2}{\partial V} E_y^2}
%   \end{split}
% \end{equation}

Again, to ensure the pressure is real, equation of the pressure is
further derived as
\begin{equation}
  \label{eq:pressure}
  \begin{split}
                                P(E_x,E_y) &= P(\tilde{E_x},\tilde{E_y})\\
                                &= \frac{\epsilon_0}{3}\braket{\epsilon_{xx}\tilde{E_x}^2 + 2
                                  \epsilon_{xy}\tilde{E_x}\tilde{E_y} +
                                  \epsilon_{yy}\tilde{E_y}^2} \mathrm{.}\\
  \end{split}
\end{equation}

Comparing equations \ref{eq:repress2} and \ref{eq:pressure}, one can
easily figure out the coefficients in \ref{eq:repress2} actually have
their physical correspondence, i.e., the polarizablity of the
material, which could be measured experimentally. With these
coefficients, the optical induced pressure is readily calculated.

\subsection{Simulation}

\begin{table*}
  \begin{ruledtabular}
                                \begin{tabular}{l c c c r}
                                  \textbf{Polarization}& $I(x)$                & $\psi(x)$           &$\Delta\phi(x)$ & $P(x)$\\
                                  \hline
                                  s-s &$2E_0^2\left(1+\cos2\delta\right)$&$\pi/2$&    $0$ &$(c_1-c_2)2E_0^2\left(1+\cos2\delta\right)$\\
                                  p-p &$2E_0^2\left(1+\cos2\delta\right)$&$0$&    $0$&$(c_1+c_2)2E_0^2\left(1+\cos2\delta\right)$\\
                                  s-p&$2E_0^2$&$\pi/4$&$-2\delta$&$2E_0^2\left(c_1+c_3\cos2\delta\right)$\\
                                  45-135&$2E_0^2$&$\delta$&$-\pi/2$&$2E_0^2\left(c_1+c_2\cos2\delta\right)$\\
                                  LCP-RCP&$2E_0^2$ &$\delta$&0&$2E_0^2\left(c_1+c_2\cos2\delta+c_3\sin2\delta\right)$\\
                                  &&&&$=2E_0^2\left(c_1+\sqrt{c_2^2+c_3^2}\sin\left[2\delta+\arctan \left(c_3/c_2\right)\right]\right)$\\
                                \end{tabular}
  \end{ruledtabular}
  \caption{Summary of the photoinduced pressure predicted by equation \ref{eq:final} for
                                various polarization conditions. $I$, $\psi$, and $\Delta\phi$ can easily be derived from
                                the interference of plane waves. $\delta=\frac{2\pi}{\lambda} x\sin\frac{\theta}{2}$ for
                                $\theta$ as in figure \ref{fig:setup}. A trigonometric identity was used to derive the
                                second form of $P(x)$ for LCP-RCP interference in order to show that for all cases
                                considered $P(x)$ can be expressed as twice the intensity of one of the initial beams
                                times the sum of a constant and a sinusoidal oscillation.} \label{tab:theory}
\end{table*}


In the holographic setups used by Saliminia and Asatryan, everything
about the two interfering beams was identical except for their
polarizations and the direction of their wave vectors. The most
general electric field produced by such interference can be written
\begin{equation}
e^{i\left(kx-\omega t\right)} \left(
\begin{array}{c}
                                \left|E_x\right| e^{i\phi_x}\\
                                \left|E_y\right| e^{i\phi_y}
\end{array}
\right) \mathrm{,} \label{eq:gene}
\end{equation}
where $\left|E_x\right|$, $\left|E_y\right|$, $\phi_x$, and $\phi_y$  are arbitrary,
real, and time (but not necessarily position) independent. Taking $E_x$ and $E_y$ from
\ref{eq:gene}, plugging them into \ref{eq:pressure}, and computing the time averages
gives

\begin{equation}
\label{eq:soln}
P \approx b_1\left|E_x\right|^2 + b_2\left|E_y\right|^2 +b_3
\left|E_x\right|\left|E_y\right|\cos{\Delta\phi},
\end{equation}


where $b_1 = \epsilon_0 \epsilon_{xx} / 6$, $b_2 =\epsilon_0
\epsilon_{yy} / 6$ and $b_3 = \epsilon_0 \epsilon_{xy} / 3$.
Defining a generalized polarization angle (applicable to elliptical polarizations)
$\psi=\arctan\left|E_y\right|/\left|E_x\right|$, equation
\ref{eq:soln} can be rewritten as
\begin{eqnarray}
P & \approx & I\left[b_1\cos^2\psi+b_2\sin^2\psi+b_3\sin2\psi\cos\Delta\phi\right] \ \ \ \ \ \label{eq:midstep}\\
& = & I(x)\left[c_1+c_2\cos2\psi(x)+c_3\cos\Delta\phi(x)\sin2\psi(x)\right]\mathrm{,} \ \
\ \ \ \  \label{eq:final}
\end{eqnarray}
where the $x$ dependence explicitly indicated in going
from equation \ref{eq:midstep} to equation \ref{eq:final}, while the
coefficients have been redefined and calculated as: $c_1 = \epsilon_0 \epsilon_{xx} / 6$, $c_2 =\epsilon_0
(\epsilon_{yy} - \epsilon_{xx}) / 6$ and $c_3 = \epsilon_0 \epsilon_{xy} / 3$. Equation \ref{eq:final} seems
intuitively reasonable since it depends on intensity, polarization, and $\Delta\phi$, the
three quantities modulation of which can cause surface relief. Taking the first spatial
derivative of Equation \ref{eq:final} yields
\begin{eqnarray}
\frac{\partial P}{\partial x}& \approx &\frac{\partial I}{\partial x}\left[c_1+c_2\cos2\psi(x)+c_3\cos\Delta\phi(x)\sin2\psi(x)\right] \nonumber\\
&&+2I(x)\frac{\partial \psi}{\partial x} \left[-c_2\sin2\psi(x)+c_3\cos\Delta\phi(x)\cos2\psi(x)\right] \nonumber\\
&&+I(x)\frac{\partial \Delta\phi}{\partial x}
\left[-c_3\sin\Delta\phi(x)\sin2\psi(x)\right] \mathrm{.} \label{eq:deriv}
\end{eqnarray}
Equation \ref{eq:deriv} cleanly separates into three independent terms
governing the pressure gradient induced by modulation of intensity,
polarization direction, and phase.  Accordingly, this model is
consistent with the idea suggested by the data that the intensity and
phase modulation terms arise from the scalar effect while the angle
modulation term is due to the vector phenomenon. Further study is
required to determine the exact mechanism by which this occurs, though
two possibilities can be identified at this point. One is that certain
bonds are excited by intensity and phase modulation and others by
polarization modulation; another is that the different illumination
conditions drive different electronic transitions in the material's
band structure. That said, this phenomenon could be due to an effect
that is entirely different from those considered above. Fortunately,
the present discussion suggests that the phenomenological approach
embodied by equation \ref{eq:deriv} could serve as the skeleton for a
complete physical model no matter what the fundamental mechanism turns
out to be.



Table \ref{tab:theory} lists the pressure functions predicted by equation \ref{eq:final}
for each of the polarization conditions tested by Saliminia and Asatryan. Despite the
widely varying initial conditions considered, the $P(x)$'s are all of roughly the same
form. Interestingly, the $c_i$'s enter in a way that evokes Saliminia's observations of
the dependence of grating amplitude's on the polarizations of the interfering beams (see
table \ref{tab:comparison}). Specifically, setting $c_2 = 3c_1$ and $c_3 = 2c_1$ begins
to approximate Saliminia's qualitative description of the observed
size ordering.
\footnote[1]{It does not matter that this scheme makes $P(x)$ negative for s-s
interference since this serves only to change the phase of the spatial oscillations in
the magnitude of of $\partial P(x)/ \partial x$}

Recalling that $E_0^2$ has a Gaussian profile (coming, as it does, from a laser), it is
fair to model the pressure as
\begin{equation}
P\sim p_1 e^{-2\left(x/p_2\right)^2}\left[p_3+\cos\left( p_4 x +p_5\right)\right]
\label{eq:presmod}
\end{equation}
where the $p_i$'s are, roughly speaking, fitting parameters (with $p_5$ included to
account for the possibility that the Gaussian intensity profile is not centered on a peak
of the modulation). In practice, the experimental setup fixes $p_2$ (the modulation
frequency) and $p_4$ (the beam radius) which are the same across all polarization
conditions. On the other hand, $p_1$, $p_3$, and to a lesser extent $p_5$ are true
degrees of freedom which can be used to fit the model to observations.

\section{Results}
\begin{table}
  \begin{ruledtabular}
                                \begin{tabular}{l c r}
                                  \textbf{Parameter}&\textbf{Meaning}&\textbf{Value}\\
                                  \hline
                                  \multicolumn{3}{c}{\textbf{Fixed Parameters}}\\
                                  $h_0$& Initial Thickness&2 $\mu$m\\
                                  $T$& Total Illumination Time&381 s\\
                                  $p_2$&Illumination Radius&57 $\mu$m\\
                                  $p_3$&Non-Oscillatory Pressure&1\\
                                  $p_4$&Modulation Frequency&$2\pi/13$ $\mu$m$^{-1}$\\
                                  $p_5$&Modulation Phase&$\pi/2$\\
                                  \\
                                  \multicolumn{3}{c}{\textbf{Free Parameters}}\\
                                  $p_1/\sigma$&Relative Pressure Strength&0.88 $\mu$m$^{-1}$\\
                                  $\sigma/\eta$&Characteristic Growth Rate&1.9$\times 10^{-3}$ $\mu$m/s\\
                                \end{tabular}
  \end{ruledtabular}
  \caption{Summary of the parameters used in constructing the fit depicted in figure
                                \ref{fig:sinemodzoom}.} \label{tab:sinemod}
\end{table}

\begin{figure}[!htbp]
  \includegraphics[width=2in]{figure/sinefigurezoom.png}
  \caption{The model's fit to a section of a surface relief profile from Saliminia's paper.}
  \label{fig:sinemodzoom}
\end{figure}

\begin{figure}[!htbp]
  \includegraphics[width=2in]{figure/saliminiagrowth.png}
  \caption{Fluence (time) dependence of maximum surface relief amplitude for the fit in figure \ref{fig:sinemodzoom}. The $x$ axis corrects an apparent typographical error in Saliminia's original paper.}
  \label{fig:salgrowth}
\end{figure}

\begin{figure}[!htbp]
  \includegraphics[width=2in]{figure/saliminiadisp.png}
  \caption{Dependence of the predicted maximum surface relief amplitude on spatial modulation frequency with all other parameters as in table \ref{tab:sinemod}. Although inconsistencies in Saliminia's reporting prevent a fit, the curve presented matches all the major qualitative features of Saliminia's graph.}
  \label{fig:saldisp}
\end{figure}


While equations \ref{eq:last} and \ref{eq:presmod} and the requirements of the numerical
methods used specify nine constants of interest, most are set by the experimental
procedure or the equations in table \ref{tab:theory}. The model parameters, both fixed
and free, are summarized in table \ref{tab:sinemod}. Of particular note are $p_1$,
$\eta$, and $\sigma$. Rewriting the pressure as $p_1 \hat{P}(x)$ allows for the recasting
of equation \ref{eq:last} as
\begin{equation}
\frac{\partial h}{\partial t} \approx \frac{\partial}{\partial x}\left(\frac{\sigma
h^3}{3\eta}\left[\frac{p_1}{\sigma}\frac{\partial \hat{P}(x)}{\partial
x}-\frac{\partial^3 h}{\partial x^3}\right]\right) \label{eq:fitmodel}
\end{equation}
which reduces the number of constants to eight and the number of fitting parameters to
two, $\sigma/\eta$ (a material property) and $p_1/\sigma$ (the relative pressure
strength). figure \ref{fig:sinemodzoom} shows the model's fit to a section of a surface
relief profile from Saliminia's paper. The parameter values used in producing this fit
are shown in table \ref{tab:sinemod}. As can be seen from figure \ref{fig:salgrowth}, the
fit thus produced also displays the correct time dependence. Finally, figure
\ref{fig:saldisp} reveals that the model can account for the observed dependence of
grating amplitude on modulation period.

% %\begin{table}\begin{center} %\begin{tabular}{|c|c|c|c|}
%\multicolumn{4}{c}{\textbf{\Large Model Parameters for the Fit in figure \ref{fig:sinemod}}}\\
%\hline
%\textbf{Parameter}&\textbf{Meaning}&\textbf{Fixed/Free} &\textbf{Value}\\
%\hline \hline
%$h_0$& Initial Thickness&Fixed&2 $\mu$m\\
%\hline
%$T$& Total Illumination Time&Fixed&381 s\\
%\hline
%$p_2$&Illumination Radius&Constrained&57 $\mu$m\\
%\hline
%$p_3$&Non-Oscillatory Pressure&Fixed&1\\
%\hline
%$p_4$&Frequency of Modulation&Fixed&$2\pi/13$ $\mu$m$^{-1}$\\
%\hline
%$p_5$&Modulation Phase&Fixed&$\pi/2$\\
%\hline
%$p_1/\sigma$&Relative Pressure Strength&Free&0.88 $\mu$m$^{-1}$\\
%\hline
%$\sigma/\eta$&Characteristic Growth Rate&Free&1.9$\times 10^{-3}$ $\mu$m/s\\
%\hline %\end{tabular} %\end{center} %\caption{Summary of the parameters used in
%constructing the fit depicted in figure \ref{fig:sinemod}.} %\label{tab:sinemod}
%\end{table}

\newpage

\begin{table}
  \begin{ruledtabular}
                                \begin{tabular}{l c r}
                                  \textbf{Parameter}&\textbf{Meaning}&\textbf{Value}\\
                                  \hline
                                  \multicolumn{3}{c}{\textbf{Fixed Parameters}}\\
                                  $h_0$& Initial Thickness&2 $\mu$m\\
                                  $T$& Total Illumination Time&381 s\\
                                  $p_2$&Illumination Radius&57 $\mu$m\\
                                  $p_3$&Non-Oscillatory Pressure&1\\
                                  $p_4$&Modulation Frequency&$2\pi/13$ $\mu$m$^{-1}$\\
                                  $p_5$&Modulation Phase&$\pi/2$\\
                                  \\
                                  \multicolumn{3}{c}{\textbf{Free Parameters}}\\
                                  $p_1/\sigma$&Relative Pressure Strength&0.88 $\mu$m$^{-1}$\\
                                  $\sigma/\eta$&Characteristic Growth Rate&1.9$\times 10^{-3}$ $\mu$m/s\\
                                \end{tabular}
  \end{ruledtabular}
  \caption{Summary of the parameters used in constructing the fit depicted in figure
                                \ref{fig:sinemodzoom}.} \label{tab:sinemod}
\end{table}

\bibliography{references}

\end{document}
