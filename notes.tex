polymer-like materials containing azobenaene units exhibit a variety
of spectacular photomechanical responses.


Recently we have proposed a gradient force model based on the dipolar
interaction of the azo-chromophores with the optically induced
electric field gradient, leading to macroscopic movement of polymeric
material and hence the grating formation.

Kolomiets was presumably the first who reported photoinduced mass
transport when studing the kinetics of 'healing' processes of
scratches inscribed on the surface of chalcogenide films. Similar
effects was reported for a-Se.

This paper presents a general model on the formation of surface relief
gratings in a series of photoresponsive amorphous thin films, such as
azobenze, a-Se, and chalcogenides.

In 1994, Hisakuni and Tanaka proposed and soon verified that
As$_{2}$S$_{3}$ displays photoinduced fluidity
\cite{hisakuni95}. Saliminia et al. (hereafter Saliminia) in 2000
discovered that thin films of As$_{2}$S$_{3}$ will respond to the
polarization of a Gaussian beam to which they are exposed by
undergoing mass transport along the polarization vector
\cite{saliminia}.

The surface relief grating was formed by illuminating
two beam interference and the surface relief shape was investigated
using the original calculation method.

Photoinduced relief deformations are very common in
azobenzene-containing materials.

These indicate that the driving force for the surface relief formation
depends on the light polarization.

The time evolution of photoinduced surface relief grating formation on
azobenzene polymer films is analyzed by ... method for fluid
mechanics.

It is generally accepted that the electric field gradient resulting
from the polarized illumination pattern triggers the dipolar defects
movement in ChG. Those dipolar defects are continuously generated by
the light and the light-induced softening...

Photoprocessed structures are
envisaged to be of high technological importance for micro or nano-
manipulation. The transformation of photo-plastic effects into
commercially viable applications will be decided from the improvement
of our knowledge on the basic mechanisms underlying the effects.


It becomes clear that pressure gradients produced in the interference
pattern are large enough to provide a driving force for mass tranport,
dependent on the intensity of the light.

Dipoles absorbing polarized light will isomerize and randomly reorient
into a position inert to the light polarization.

Similar behaviors are observed in other types of materials which also
exhibit a strong photoinduced change in the conformation at the
molecular level.

We relate the formation of SRGs to a photo-induced pressure. This
pressure and, hence, the height of a SRG depend on the laser
polarization and the susceptibility of the material.

Such a study will improve our general understanding of the response of
azonbenze-containing materials to light irradiation and is, therefore,
of high practical importance.

Different theories exist which try to explain the formation of surface
relief gratings.

All our experimental results are in good agreement with the gradient
force model.

We found that all experimental results can be explained with the
gradient force model, and we suggest using this model to be an unified one.

Chalcogenide glasses and amorphous layer of As$_2$S$_3$, AsSe, GeAsSe
and similar compositions reveal well-known photo-induced structural
transformation effects within the amorphous phase.

SRGs are induced in different holographic schemes of recording using
near-band-gap light.

It is found that the kinetics of SRG formation depends upon incident
polarization.

Lateral mass transport phenomena under the optical gradient force.

As the phenomena depends on the light intensity and light polarization
state.

Surface relief formation at holographic recording on amorphous
selenium films was investigated.

A number of studies have been carried out on photoinduced structural
transformation in amorphous selenium during the past few
years. Several new ideas were introduced for the explanation, like
photo-crystallization and also the general mechanism in
light-sensitive chalcogenide glasses.

Further steps include a structural re-arrangement under the condition
of photo-induced fluidity.

Surface grating formation was previously also observed in different
classes of anzobenzene polymers. Several models have been proposed to
account for laser-induced surface deformation in these materials,
based on laser ablation, photo-driven migration of the polymer chains,
optical field gradients and pressure gradient effects resulting from
the spatial variation of trans and cis isomer having different
volumes. Volume expansiion was also observed in amorphous chalcogenide
films under the influence of electron beams or light.

Our proposed model is based on the following considerations. It was
demonstrated...

Surface relief formation during holographic recording in a-Se layers
was demonstrated and attributed to the reversible volume contraction
due to the localized structural ordering of...

A driving force of photoinduced surface relief formation is examined
by considering the complex electric susceptibility of the absorbing
medium. It is introduced into a fluid mechanics model established for
the viscous fluid layer on a flat substrate. As a result, a
theoretical model for the PSR formation covering thoroughly form the
origin of the driving force to the dynamic process forming the surface
relief is established. The experimental results reported in former
literature are also reviewed using this model and the result suports
strongly our assumption: driving force derived from the interaction
between the absorbing medium and optical electric field accts as a
body force on the optically plasicized azopolymer.

In 1995, it was discovered that thin films of an azobenzene containing
polymer form a surface modulation under the irradiation with an
interference pattern of a lser beam.

In the past decade, surface relief gratings (SRGs) have attracted attention
from the viewpoint of mass transfer. SRGs formed on
azobenzene-functionlized polymer films have attracted the interest of
many researchers. SRGs formed on films composed of various amorphous
polymers.

Although many experimental and theoretical studies have been reported
in relation to SRG formation, the mechanism of its formation is not
yet fully understood.

In this study, we describe the SRG formation by diffusive mass
transfer caused by photo-induced pressure arising from realignment of
photo-induced dipoles.

A number of studies have been carried out on photoinduced structural
transformation in amorphous thin films, made from various materials,
such as azobenze-containing material, a-Se, and different
chalcogenides, like Arsenic Sulfide (As$_2$S$_3$), Arsenic selenide
(As$_2$Se$_3$), etc.

It has long been known that photoinduced processes can lead to mass
transport in a variety of systems generating surface relief.  A
variety of systems exhibit photoinduced expansion and or contraction.

While it has long been known that thin arsenic sulfide
(As$_{2}$S$_{3}$) films expand when exposed to above-bandgap light,
the mechanism by which this occurs is still open to debate
\cite{igo74, hegedus, ganjoo}. Two fairly recent discoveries reveal
that the situation is far more complex than it first appears. First,


The second major development was the discovery by  Salminia's group also exposed films to several
interference patterns with intensity gradients and to others with
modulated polarization and observed the formation of surface relief
gratings of a size comparable to the ``giant'' photoexpansion in both
cases. Although surface relief gratings had been observed in the past,
those early studies reported much smaller features \cite{galstyan}. In
2005, Asatryan et al.  (hereafter Asatryan) repeated Saliminia's
experiment at much lower intensities and found a similar result for
polarization modulation but did not observe grating formation under
intensity contrast \cite{asatryan05}. Saliminia and Asatryan's results
are summarized in table \ref{tab:comparison}.

After addressing and explaining the apparent contradictions between
Saliminia and Asatryan's data, this paper presents a phenomenological
model describing surface relief formation as incompressible flow
driven by a photoinduced pressure and damped by surface
tension. Despite the fact that both polarization and intensity
modulation may lead to this effect, a single expression for the
photoinduced pressure, derived without any assumptions about the
microstructure of As$_2$S$_3$, is shown to be sufficient. Finally,
results of the model are considered.

However the mechanism was not
well understood until now. , although several models
have been proposed by some research groups that attempt to explain the
mechanism. Therefore a novel theoretical model was proposed for
photoinduced mass transport and numerical analysis was performed.
